%\documentclass[a4paper, 12pt, twoside, openright, titlepage, enabledeprecatedfontcommands]{scrbook}
\documentclass[a4paper, 12pt, twoside, openright, titlepage]{book}
\usepackage[english]{babel}
\usepackage[utf8]{inputenc}
\usepackage[T1]{fontenc}
%\usepackage{lmodern}
\usepackage{scrlayer-scrpage} % stili pagina per il frontespizio

\usepackage{booktabs}
\usepackage{caption}
\usepackage{amssymb}
\usepackage{subfig}

\usepackage[beramono, pdfspacing, dottedtoc,linedheaders]{classicthesis}
%\usepackage{arsclassica}
\usepackage{amsmath, amsfonts} % AMS Math Package
\usepackage{amsthm} % Theorem Formatting

\usepackage{amssymb, verbatim,mathtools,needspace,enumitem,etoolbox,graphicx, physics,microtype,afterpage,bigints,gensymb,tabularx, comment}

\usepackage[top=2.5cm, bottom=2.5cm, inner=4cm, outer=4cm, right=2.5cm, centering]{geometry}
\usepackage{emptypage}
%\usepackage{fancyhdr}



% Interlinea
\linespread{1.5}


\usepackage{csquotes} % per le citazioni "in blocco"
\usepackage[square]{natbib} % bibliografia con pacchetto biblatex (https://ctan.org/pkg/biblatex?lang=en)
\bibliographystyle{apsrev4-1}
%\bibliographystyle{apsrev4-2-author-truncate}
%\bibliographystyle{plain}
%\bibliographystyle{apacite}
\setcitestyle{numbers}
\appto{\bibsetup}{\raggedright}

\usepackage{titlesec} % per la formattazione dei titoli delle sezioni, capitoli etc.
\usepackage{float} % per il posizionamento delle immagini

\usepackage{listings} % per il codice di programmazione
% Fonte https://en.wikibooks.org/wiki/LaTeX/Source_Code_Listings. Per la lista di sintassi riconosciute.
\renewcommand{\lstlistingname}{Code}% Listing -> Codice
\usepackage{xcolor}  % stile del codice
\usepackage{hyperref}

\graphicspath{{../images}}


%%fancy stuff
%%\pagestyle{fancy}
%
%% Redefine \chaptermark and \sectionmark to remove prefixes
%\renewcommand{\chaptermark}[1]{\markboth{\ #1}{}}
%\renewcommand{\sectionmark}[1]{\markright{\thesection $\,\,$ \ #1}}
%
%\fancyhf{} % Clear all header and footer fields
%\fancyhead[LE,RO]{\small \thepage} % Page number on the outer corners
%\fancyhead[RE]{\small \spacedlowsmallcaps \leftmark} % Chapter title on the right of even pages (LE)
%\fancyhead[LO]{\small \spacedlowsmallcaps \rightmark} % Section title on the left of odd pages (RO)
%\setlength{\headsep}{11pt}
%

\title{Annual Progression Review}
\author{Cecilia Maria Fabbri}
\date{1 Steptember 2025}	

\begin{document}
\bibliographystyle{plain}
% Frontespizio

\maketitle


% Fine frontespizio
%\thispagestyle{empty}
%\clearpage

\frontmatter

\tableofcontents
%\thispagestyle{empty}

%\listoffigures
%\thispagestyle{empty}

%\listoftables

%\clearpage

\begin{chapter}{Literature Review}
- intro gw super quick -> we have a signal! (add if you have space: good candidates to detect are binary systems, detectors, current catalogue description)
- data analysis for single event
- data analysis for populations
- sbi basics, single event
- sbi population
- growing pains (accuracy requirements)


Gravitational waves are perturbations of spacetime generated by mass distributions with non-null second derivative of the quadrupole mass-moment. 
When emitted, they travel through the Universe carrying pristine information on the event producing them.

The LIGO-Virgo-KAGRA collaboration uses a system of intereformeters to detect perturbations of spacetime and they build up catalogues of gw events.
The latest complete catalogue of gw transient events is gwtc-3 and contains x signals generated by binaries of nss and bhs.
Since the amplitude of the wave and its evolution depend on the properties of the system, with data analysis we can infer the properties of the system producing the detected wave.

-The following part is not too much impo for my work, so don't deepen it too much-
The first challenge is the detection of signals of astrophysical origin. 
If we consider a binary with ... (add some properties such as mass etc), then the relative change in the length of arms of the interferometer is $\sim 10^{-21}$, which, for a typical intereferometer arm of $5 km$, is ... cms, i.e. x orders of magnitude smaller than the size of an atom (check).
With such a small amplitude of astrophysical signal, the output signal is dominated by noise. 
To assess how confident we are that a segment of data contains astrophysical signal, typically one defines a metric such as the signal to noise ratio (SNR) or the false alarm rate (FAR). 
Then one sets a treshold on the metric to decide when to analyze data as containing an event.
(The procedure LVK uses to detect signals strongly relies on accurate modelling possible gravitational-wave signals embedded in the noise. 
(This is useful just to explain the type of catalogue we have)).


Once an astrophysical signal is detected in a data segment, the following step is to analyze the data and with the goal of inferring the properties of the event producing that signal. (like that it seems that we infer only one event, while instead is a distribution over properties given a signal.)



The output of the detector is $d=h(\theta)+n$, with h(t) the astrophysical signal, 
and n(t) the noise realization at that time, which is usually louder than the gravitational-wave signal. (already said)
$h(t)$ is assumed to be deterministic, and it depends on the binary properties.
We define  



---------

(Noise is not deterministic, which means we don't care its value at t but we care about is statistical properties -> better to go in the fourier domain. We assume it is gaussian, stochastic and stationary. This means it is completely defined by its autocorrel function and its average)
----------

in analyses we need models to evaluate the likelihood -> expensive





\end{chapter}

\begin{chapter}{Completed Work}
a descripiton of work completed to date
- small project description

Transfer learning is a series of techniques used in machine learning to transfer knowledge from a neural network to another.
Each time one modifies a neural network, for instance changing the model assumed in the analysis, the neural network needs to be trained again. 
The goal of transfer learning is to fine tune a new training run using some knowledge acquired during a previous training.
These strategies allow to reduce training time, and, more importantly, could even improve the results respect to training the new run without any information from previous trainings. 




\end{chapter}

\begin{chapter}{Work Plan}
a plan of work for the next 12 months
- pop growing pains for sure
- we talked about using astrophysical models as pop models. We need to understand better. Surely something more astro
- another idea is to keep developing the current project (NR simulations, beyond GR models)
\end{chapter}

\begin{chapter}{Personal Development Plan}
Copy of personal development plan summary and a statement of progress made towards those training goals
\end{chapter}

\appendix
%
%\pagestyle{fancy}
%
%\fancyhf{} % Clear all header and footer fields
%\fancyhead[LE,RO]{\small \thepage} % Page number on the outer corners
%\fancyhead[RE]{\small \spacedlowsmallcaps \leftmark} % Chapter title on the right of even pages (LE)
%\fancyhead[LO]{\small \spacedlowsmallcaps \leftmark}
%

\backmatter
%%BIBLIOGRAPHY
\bibliography{thesis}

\end{document}
